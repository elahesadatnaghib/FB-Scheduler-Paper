\documentclass[12pt]{article}
\usepackage[utf8]{inputenc}
\usepackage[english]{babel}
 
\usepackage{multicol}
\usepackage{amsmath}
\usepackage{amssymb}
\usepackage{amsfonts}
\usepackage{amsthm}
\usepackage[colorlinks=true, allcolors=blue]{hyperref}

\topmargin 0.0cm
\oddsidemargin 0.2cm
\textwidth 16cm 
\textheight 21cm
\footskip 1.0cm

\newtheorem{prop}{Proposition}
\theoremstyle{definition}
\newtheorem{defn}{Definition}

\DeclareMathAlphabet{\pazocal}{OMS}{zplm}{m}{n}
\DeclareMathOperator*{\argmin}{arg\,min}

\begin{document}
\title{Proposing a feature-based scheduler for the Large Synoptic Survey Telescope}
\maketitle
%\begin{multicols}{2}
\section{Introduction} \label{sec:intro}

\begin{itemize}
\item Challenges of scheduling a telescope in general
\item Review Spike (Hubble space telescope scheduler) \cite{johnston1994spike}, Look-ahead technique for space telescope scheduling \cite{sadeh1991look}, minimizing conflicts \cite{minton1992minimizing}
\item Transition from space telescope to ground-based telescope : (1) how constraints are different, (2) why there hasn't been much work for ground based scheduling \textit{(if that's true)}(not fast enough not expensive enough, accessible for human intervention, narrow ranged mission objectives)
\item Review the scheduler of recent comparable ground-based telescopes (E-ELT, and ?, ?)
\item LSST specifications and general desirables, why lsst needs an automated scheduling approach (it's expensive, fast, and has conflicting objectives)
\item Characteristics of a scheduler for a ground-based telescope (computational aspects, controllable for science measures, responsive to the weather's unpredictable changes)
\item Review Opsim scheduler \cite{delgado2016lsst}
\item What Feature-based scheduler has to offer
\end{itemize}
 
 
\section{Scheduling framework}\label{sec_SM}
\textit{This section is planned to be written in the full extent, if it became unnecessarily long, we can always use it for the other paper, and summarize it here}

To run a multi-mission telescope such as LSST, there are a certain number of capabilities that the scheduler is supposed to offer: Controllability, adjustability, and recoverability.
\begin{itemize}
\item \textbf{Controllability}: For the choice of information that a scheduler uses to make a decision, there has to be a meaningful correlation with the high-level mission objectives. In other words, a scheduler has to demonstrate that the mission is controllable by means of the information (including mechanical system variables and constraints, environmental variables, and disturbances), that it takes as an input.

\item \textbf{Adjustability}: For a complex, multi-objective mission it is common that the high level objectives are required to be adjusted in the middle of the operational period, adjustments take place according to a new set of desirables, or based on a feedback provided by the study of system's performance. Regardless of the reason for adjustments, a scheduler must offer a systematic flexibility to accordingly respond to the adjustments in the mission with a reasonable computational cost, and preferably no expert intervention. Hand-tuned scheduling strategies for instance, does not offer a full adjustability attribute.

\item \textbf{Recoverability}: Presence of unpredictable factors in the decision process of operating ground-based telescopes are due to the natural stochastic processes (such as weather), and complexity of the mission and the facility that brings randomness to the decision procedure. Unscheduled downtimes, filter breakdown, are examples of the many unpredictable survey interruptions. On the other hand, there are also inherently predictable interruptions, such as maintenance downtimes, and cable winding that again due to complexity of the mechanical system, it is not computationally affordable and/or valuable to keep track of. Therefore, they would have to be considered as stochastic variables as well. Given all the stochastic factors, a scheduler is required to be able to output a fast alternative once a previously unpredictable variable took a value, as well as occurrence of an interruption, for which, the scheduler is expected to return to its normal behavior, as fast as possible. Pre-calculated sequence of decisions for instance, lack the recoverability attribute and strategies with heavy look-back or look-ahead are not fast recoverables. 
\end{itemize}

In this study we propose a scheduler that is shown to be \textit{controllable} based on the simulation results in Section \ref{sec_sim_cont}, \textit{adjustable}, because of the structure that offers an explicit derivation of the design elements from high level objectives, and finally is \textit{recoverable} due to the Markovian representation of the decision process.

\subsection{Markovian representation}
\begin{defn} 
Let $s_{t^n}$ be a random vector, representing state of the system at $t^n$, then the process of  $<S, A, \pazocal{T}, R>$  is Markovian if and only if,\\
\begin{equation*}
\forall n\geq 0 ~~~~~~~p(s_{t^{n+1}} | s_{t^n}) = p(s_{t^{n+1}} | s_{t^n}, s_{t^{n-1}},\dots, s_{t^0})  
\end{equation*}
Where, $p(s_{t^{n+1}} | s_{t^n})$  is conditional probability of the system's state at $t^{n+1}$ given the system's state at $t^n$, and the right hand side of the equality is the conditional probability of the same random vector given the full sequence of system's state from the initial state $s_{t^0}$, to the most recent state $s_{t^n}$.
\end{defn}

\begin{defn}
An action $a_{t^n}$ is admissible if $a_{t^n} \in A_{t^n}$ and is progressively measurable with respect to $\sigma\{s_{t^m}:~m \leq n\}$. Where $\sigma\{s_{t^m}:~m \leq n\}$, is the sigma algebra, generated by the Markovian process $s_{t^{(.)}}$.
\end{defn}

Therefore, admissible actions at $t^n$ can be written as a function of $s_{t^n}$, which is called \textit{policy},
\begin{equation*}
a_{t^n} = \Pi(s_{t^n})
\end{equation*}

\begin{defn}
An optimal policy $\Pi(s_{t^n})$ is a solution to the following optimization problem,
\begin{equation}\label{equ_opt1}
\begin{aligned}
& \underset{\Pi}{\text{maximize}}
& & E[\sum_{n=1}^{N} R(s_{t^{n-1}}, s_{t^n}) | \Pi] 
\end{aligned}
\end{equation}
\end{defn}

\begin{prop}
If the system of  $<S, A, \pazocal{T}, R>$  is Markovian , then there exist an optimal policy, and it can be written as follows,
\begin{equation}\label{equ_opt_pol}
\Pi^*(s_{t^n}) = \argmin_{a_{t^n}} E[\Phi(s_{t^{n+1}}) | a_{t^n}]
\end{equation}
\end{prop}

Thus the solution space of the optimization problem (\ref{equ_opt1}), can be reduced from search over policies to search over $\Phi(s_{t^{(.)}})$ functions,

\begin{equation}\label{equ_opt2}
\begin{aligned}
& \underset{\Phi}{\text{maximize}}
& & E[\sum_{n=1}^{N} R(s_{t^{n-1}}, s_{t^n}) | \Phi] 
\end{aligned}
\end{equation}

\subsection{Feature space approximation}
For a decision that is inherently time dependent, such as scheduling an observation, only a maximal definition of system's state yields a perfect Marovian Process. A maximal definition of the system's state includes all possible sequences of decisions, which is practically impossible to keep track of, due to both memory and computational limitations. In particular, LSST requires a sequence of about 1000 decisions at a night that requires a state space of $N_{f}^1000$, that clearly is not tractable even if it is for a single average night.

\subsection{Policy function approximation}
Problem (\ref{equ_opt2}) is defined as a search over a space of functions, therefore it is an infinite dimensional optimization problem. To be able to numerically compute the $\Phi$ function, we propose a parametrized function approximation with linear dependency on the vector of approximate state,

\begin{equation*}
 \pazocal{L}_{\theta}(\tilde{s}_{t^n}) := \sum_i \theta_i \tilde{s}^i_{t^n} ~~~and~~~\Phi(\tilde{s}_{t^n}) \approx  \pazocal{L}_{\theta}(\tilde{s}_{t^n})
\end{equation*}

Where $\theta$ is the vector of free parameters that characterize, $\pazocal{L}(.)$, the approximate linear function. With this approximation, the space of search is reduced from the space of functions to the finite dimensional vector space. This approximation substitutes the original optimal policy (\ref{equ_opt_pol}) with the following approximate policy,

\begin{equation*}\label{equ_approx_pol}
\tilde{\Pi}^*(s_{t^n}) = \argmin_{a_{t^n}} E[ \pazocal{L}_{\theta^*}(\tilde{s}_{t^n+1}) | a_{t^n}]
\end{equation*}

Where $\theta^*$ is a solution to the following optimization problem,

\begin{equation*}\label{equ_opt3}
\begin{aligned}
& \underset{\theta}{\text{maximize}}
& & E[\sum_{n=1}^{N} R(s_{t^{n-1}}, s_{t^n}) | \theta] 
\end{aligned}
\end{equation*}


\begin{itemize}
\item Optimality conditions (history and future independence)
\item Measurability conditions (on the sigma algebra of the decision making process)
\item Optimal structure (min cost(i,f)), Linear/quadratic structure of cost
\item Optimizer
\item Challenges to reach optimality (features would not contain all the information, cost function is structured, optimization is non-linear)
\end{itemize}


\section{Region-Dependent Basis function } \label{subsec-BF}
\textit{Illustrate that a number of simply designed basis functions can provide the scheduler with fine and tunable behavior (also in the abstract)}\\
\subsection{Basis Functions common values}
\textit{this section is going to be around 8 pages long, we might as well publish the details in a tech report and then cite it here}\\
Different regions of the sky require different visit and revisit constraints and priorities. Therefore, to evaluate a cost of operation for fields of each region there are different basis functions, that at the same time are required to be scaled appropriately, so that there will be no priority given to a certain region only due to the design of the basis functions. In this section we introduce a set of functions that are shared among the basis functions of the different regions, to provide a similar base value for all of the regions.
 
\section{Scheduler optimization}
\textit{Demonstrate how much the optimization part matters with the results of applying the solutions of earlier iterations}
\subsection{Differential Evolution}

\subsection{Objective Function}

\section{Simulation Results}\label{sec_sim}
\textit{count diagrams, Open shutter fraction, meridian observation}
\subsection{controlability check}\label{sec_sim_cont}
\section{Discussion}
\textit{In line with the amount of telescope's time/money it would save}




\medskip

\newpage 
\bibliographystyle{unsrt}
%\bibliography{/Users/elahesadatnaghib/Dropbox/Graduate/Research/Princeton/Publications/references}
\bibliography{/Users/enaghib/Dropbox/Graduate/Research/Princeton/Publications/references}

%\end {multicols}
\end{document}




